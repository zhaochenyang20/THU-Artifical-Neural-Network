\documentclass{article}

\usepackage{arxiv}
\usepackage[utf8]{inputenc} % allow utf-8 input
\usepackage[T1]{fontenc}    % use 8-bit T1 fonts
\usepackage{hyperref}       % hyperlinks
\usepackage{url}            % simple URL typesetting
\usepackage{booktabs}       % professional-quality tables
\usepackage{amsfonts}       % blackboard math symbols
\usepackage{nicefrac}       % compact symbols for 1/2, etc.
\usepackage{microtype}      % microtypography
\usepackage{graphicx}
\usepackage{subcaption}
\usepackage{natbib}
\usepackage{doi}
\usepackage{ctex}
\usepackage{multirow}
\usepackage{booktabs}
\usepackage{float}
\title{人工神经网络 MLP NUMPY}
\date{\today}
\author{
	赵晨阳\\
	2020012363\\
	清华大学计算机系\\
	\texttt{zhaochenyang20@gmail.com}
}

\renewcommand{\headeright}{人工神经网络}
\renewcommand{\undertitle}{}
\renewcommand{\shorttitle}{MLP Numpy}
\begin{document}
\maketitle

\section{基本实验}
\subsection{参数设定}
基础实验部分采用的超参数如下所示,注意到部分网络架构在该超参数设定下可能无法成功训练。
\begin{verbatim}
	config = {
    'learning_rate': 1e-2,
    'weight_decay': 0,
    'momentum': 0.0,
    'batch_size': 10,
    'max_epoch': 50,
    'disp_freq': 50,
    'test_epoch': 1
}
\end{verbatim}

\subsection{图例说明}

如图~\ref{fig:1} 所示,实验结果图表标题分别展示了 test\_loss 与 train\_loss。\\
每条图例曲线命名规则为:\\
\textbf{\{hidden layer num\}\_\{activation function name\}\_\{loss function name\}\_\{learning rate\}\_\{weight decay\}\_\{batch size\}}\\
倘若没有尾缀的三个超参数,则表示采用默认参数设定。

\begin{figure}[htbp]
	\centering
	\begin{subfigure}{0.475\textwidth}
		\centering
		\includegraphics[width=1\textwidth]{../pics/单层实验-test_loss.png}
		\caption{单隐藏层实验 test loss}
	\end{subfigure}
	\begin{subfigure}{0.475\textwidth}
		\centering
		\includegraphics[width=1\textwidth]{../pics/消减率_2_Gelu_HingeLoss_train_loss.png}
		\caption{单隐藏层实验 test accuracy}
	\end{subfigure}
\caption{图例说明}
\label{fig:1}
\end{figure}

\subsection{零隐藏层实验}

对于零隐藏层网络,分别使用三种损失函数,训练过程和训练结果如图 ~\ref{fig:2} 所示。可以发现如果只有一层线性层的话,没有隐藏层,则无法达到预期的学习效果。

\begin{figure}[htbp]
	\centering
	\begin{subfigure}{0.475\textwidth}
		\centering
		\includegraphics[width=1\textwidth]{../pics/零层实验-test-loss.png}
		\caption{零隐藏层 test loss}
	\end{subfigure}
	\begin{subfigure}{0.475\textwidth}
		\centering
		\includegraphics[width=1\textwidth]{../pics/零层实验-train-loss.png}
		\caption{零隐藏层 train loss}
	\end{subfigure}
	\begin{subfigure}{0.475\textwidth}
		\centering
		\includegraphics[width=1\textwidth]{../pics/零层实验-test-acc.png}
		\caption{零隐藏层 test accuracy}
	\end{subfigure}
	\begin{subfigure}{0.475\textwidth}
		\centering
		\includegraphics[width=1\textwidth]{../pics/零层实验-train_acc.png}
		\caption{零隐藏层 train accuracy}
	\end{subfigure}
	\caption{零隐藏层实验效果}
	\label{fig:2}
\end{figure}

\subsection{单隐藏层实验}
增加单层隐藏层,隐藏层神经元数设置为 100,遍历三种损失函数和三种激活函数进行对比实验。其结果如图 \ref{fig:3} 所示。

\begin{figure}[htbp]
	\centering
	\begin{subfigure}{0.475\textwidth}
		\centering
		\includegraphics[width=1\textwidth]{../pics/单层实验-test_loss.png}
		\caption{单隐藏层 test loss}
	\end{subfigure}
	\begin{subfigure}{0.475\textwidth}
		\centering
		\includegraphics[width=1\textwidth]{../pics/单层实验-train_loss.png}
		\caption{单隐藏层 train loss}
	\end{subfigure}
	\begin{subfigure}{0.475\textwidth}
		\centering
		\includegraphics[width=1\textwidth]{../pics/单层实验-test_acc.png}
		\caption{单隐藏层 test accuracy}
	\end{subfigure}
	\begin{subfigure}{0.475\textwidth}
		\centering
		\includegraphics[width=1\textwidth]{../pics/单层实验-train_acc.png}
		\caption{单隐藏层 train accuracy}
	\end{subfigure}
	\caption{单隐藏层实验效果}
	\label{fig:3}
\end{figure}

\subsubsection{不同损失函数函数对比}

在单隐藏层的前提下,固定激活函数为 Sigmoid,遍历三种损失函数,结果如图 \ref{fig:4} 所示。\\
分析结果可以发现,(1)就准确率而言,在测试集和训练集上均是 Hinge 效果最好而 Euclidean 效果最差;(2)就损失值而言,Hinge、SoftmaxCrossEntropy 和 Euclidean 计算出的绝对值存在着明显的差异,损失值无法横向比较,然而对比损失值下降速度,能够观察到 Hinge 收敛效果与收敛速度最为显著;(3)就收敛速度而言,Hinge 收敛曲线斜率最高,收敛最快,而 Euclidean 收敛最慢;(4)就训练时间而言,Hinge 耗时为 
10m 49s,Euclidean 耗时为 10m 45s,SoftmaxCrossEntropy 耗时也为 10m 45s,三者并无显著差异。

\begin{figure}[htbp]
	\centering
	\begin{subfigure}{0.475\textwidth}
		\centering
		\includegraphics[width=1\textwidth]{../pics/单层损失函数test_loss.png}
		\caption{单隐藏层遍历损失函数对比 test loss}
	\end{subfigure}
	\begin{subfigure}{0.475\textwidth}
		\centering
		\includegraphics[width=1\textwidth]{../pics/单层损失函数train_loss.png}
		\caption{单隐藏层遍历损失函数对比 train loss}
	\end{subfigure}
	\begin{subfigure}{0.475\textwidth}
		\centering
		\includegraphics[width=1\textwidth]{../pics/单层损失函数test_acc.png}
		\caption{单隐藏层遍历损失函数对比 test accuracy}
	\end{subfigure}
	\begin{subfigure}{0.475\textwidth}
		\centering
		\includegraphics[width=1\textwidth]{../pics/单层损失函数train_acc.png}
		\caption{单隐藏层遍历损失函数对比 train accuracy}
	\end{subfigure}
	\caption{单隐藏层遍历损失函数}
	\label{fig:4}
\end{figure}

比较 Hinge、SoftmaxCrossEntropy 和 Euclidean 的函数形式可见,Hinge 损失函数的计算形式最为复杂,效果出人意料更好。然而之前接触过的分类任务更多使用 SoftmaxCrossEntropy。没有选择使用 Hinge 的原因可能是 Hinge 还还存在超参数 Margin 的缘故。对于 Margin 参数的 tune 也能作为实验目标。


\subsubsection{不同激活函数对比}

在单隐藏层的前提下,固定损失函数为 Euclidean,遍历三种激活函数,结果如图 \ref{fig:5} 所示。

\begin{figure}[htbp]
	\centering
	\begin{subfigure}{0.475\textwidth}
		\centering
		\includegraphics[width=1\textwidth]{../pics/单层激活函数test_loss.png}
		\caption{单隐藏层遍历激活函数对比 test loss}
	\end{subfigure}
	\begin{subfigure}{0.475\textwidth}
		\centering
		\includegraphics[width=1\textwidth]{../pics/单层激活函数train_loss.png}
		\caption{单隐藏层遍历激活函数对比 test loss}
	\end{subfigure}
	\begin{subfigure}{0.475\textwidth}
		\centering
		\includegraphics[width=1\textwidth]{../pics/单层激活函数test_acc.png}
		\caption{单隐藏层遍历激活函数对比 test loss}
	\end{subfigure}
	\begin{subfigure}{0.475\textwidth}
		\centering
		\includegraphics[width=1\textwidth]{../pics/单层激活函数train_acc.png}
		\caption{单隐藏层遍历激活函数对比 test loss}
	\end{subfigure}
	\caption{单隐藏层遍历激活函数}
	\label{fig:5}
\end{figure}

分析结果可以发现,(1)就准确率而言,ReLU 结果略微高于 GeLU,而 Sigmoid 的表现与二者有一定差距;(2)就损失值而言,ReLU 损失值略微低于 GeLU,而 Sigmoid 损失值显著大于前两者,且损失值衰减速率基本与准确率的上升速率符合;(3)就收敛速度而言,ReLU 与 Gelu 收敛速度相当,而 Sigmoid 收敛最慢,且收敛值最低;(4)就训练时间而言,Sigmoid 耗时为 10m 45s,Relu 为 10m 33s,Gelu 为 11m 14s,考虑到 Relu 最为简单的函数形式,具有最快的训练速度符合预期。而 Gelu 的实现方式基于微分的极限定义,计算复杂度最高,也符合预期。

对比图 \ref{fig:5},考虑到三者的函数定义。
\begin{itemize}
    \item $Gelu(x)=x \times P(X<=x)=x \times \phi(x), x \sim N(0,1) \approx  0.5 \times x\left(1+\tanh \left[\sqrt{\frac{2}{\pi}}\left(x+0.044715 x^3\right)\right]\right)$
    \item $Relu(x)=x^{+}=\max (0, x)$
    \item $Sigmoid(x)=\frac{1}{1+e^{-x}}=\frac{e^x}{e^x+1}=1-Sigmoid(-x)$
\end{itemize}

Sigmoid 和 GeLU 光滑且连续可微的函数;而 GeLU 被誉为平滑的 ReLU,二者图像和导数值相近,在损失值为正时,具有相对稳定的导数。然而,Sigmoid 的效果一贯较差是深度学习领域研究者多年得出的一贯结论。具体而言,Sigmoid 在损失值较大时,实际上倒数值偏小,梯度消失非常严重。而我们观察到对 Euclidean 损失而言,初始的损失值就是偏大的,导致 Sigmoid 一直处于倒数值偏小的区间,严重限制了收敛效率与最终的收敛效果。

\subsubsection{过拟合问题}

实际上在本次实验中并未出现明显的过拟合现象,基本上实验结果保证了 loss 和 accuracy 负相关。然而,这一现象实际上并不绝对,在复杂模型处理复杂任务的情况下,往往会出现严重的过拟合。此外,即便是将 weight decay 设置为 0,也没有出现过拟合,实际上预示了 weight decay 参数可能对实验效果有着负面影响,而这在 \ref{paragraph:1} 一节得到了印证。

\subsection{双隐藏层实验}

增加双层隐藏层,隐藏层神经元数分别设置为 100 和 100,遍历不同损失函数和激活函数进行实验,得到的训练过程和训练结果分别如图 ~\ref{fig:6} 所示。相比同样单隐藏层训练结果,双隐藏层的网络在测试集、训练集上的表现并未比单隐藏层存在显著优势,甚至在许多模型设定下表现不如单隐藏层。出于常识,双隐藏层网络比单隐藏层网络拥有更多的参数,似乎该具有更强的学习能力。\\
然而,这个结论并不必然。一方面,小模型并不意味着更弱的表现能力,知识蒸馏领域的研究者便希望能够用更小的模型达到更好的效果;另一方面,在这次实验中,双层的参数复杂度上升,需要的训练资源和训练技巧相应更多,然而实验过程并没有为模型添加 dropout 层,也没有参考 RNN 的实现而为双层模型使用 residual connection。另外,实验没有尝试调整 momentum 参数,而将其固定为 0.0,这等效于使用了最为原始的 SGD 优化器,进一步致使模型难以体现出增加隐藏层带来的表达力优势。最后,考虑到十分类任务本身已经足够简单,即便是最简单的单隐藏层 MLP 也有了几乎完全正确的结果,可见单层模型的表达能力已经足够强大;双层模型并不能在模型表达力上有大幅突破,反而因为训练难度、模型复杂度和对训练技巧的更高需求丧失了潜在的优势。\\
综上所述,双隐藏层的结果不如单隐藏层,能够得到充分的理解。
\begin{figure}[htbp]
	\centering
	\begin{subfigure}{0.475\textwidth}
		\centering
		\includegraphics[width=1\textwidth]{../pics/双层实验-test_loss.png}
		\caption{双隐藏层 test loss}
	\end{subfigure}
	\begin{subfigure}{0.475\textwidth}
		\centering
		\includegraphics[width=1\textwidth]{../pics/双层实验-train_loss.png}
		\caption{双隐藏层 train loss}
	\end{subfigure}
	\begin{subfigure}{0.475\textwidth}
		\centering
		\includegraphics[width=1\textwidth]{../pics/双层实验-test_acc.png}
		\caption{双隐藏层 test accuracy}
	\end{subfigure}
	\begin{subfigure}{0.475\textwidth}
		\centering
		\includegraphics[width=1\textwidth]{../pics/双层实验-train_acc.png}
		\caption{双隐藏层 train accuracy}
	\end{subfigure}
	\caption{双隐藏层实验效果}
	\label{fig:6}
\end{figure}

此外,注意到实际上双隐藏层时,某些模型设定并不能成功训练。考虑到训练失败的设定是 2\_Sigmoid\_SoftmaxCrossEntropy 和 2\_Sigmoid\_Euclidean,而 2\_Sigmoid\_Hinge 成功训练。\\
叠加双层的 Sigmoid 作为激活函数,可能会存在非常严重的梯度消失问题,而只有表达力更强的 Hinge 能够得到训练,这也符合预期。\\
当然,这里所指的不能成功训练是基于基础实验的超参数设计而言的,实际上将学习率从 1e-2 提高到 1 即可让 2\_Sigmoid\_SoftmaxCrossEntropy 成功训练;这也符合预期,因为提高学习率也是对抗梯度衰减的一大方法。基于此,扩展实验中对学习率的调节便是基于 2\_Sigmoid\_SoftmaxCrossEntropy 而展开。而 2\_Sigmoid\_Euclidean 设定即是调整了学习率,依然难以训练成功。据此推断,没有使用 residual connection、dropout 且没有调整 momentum 的情况下,层数越多 Sigmoid 导致的梯度消失问题越严重,劣势将进一步被放大。

\section{扩展实验}

\subsection{学习率影响}

保留基础实验的参数设定,模型选取 1\_Sigmoid\_EuclideanLoss(图 \ref{fig:7}) 与 2\_Sigmoid\_Softmax(图 \ref{fig:8}),学习率选取为 \{1e-4, 1e-3, 1e-2, 1e-1, 1\},展开学习率对比实验。

\begin{figure}
	\centering
	\begin{subfigure}{0.475\textwidth}
		\centering
		\includegraphics[width=1\textwidth]{../pics/学习率_1_Sigmoid_EuclideanLoss_test_loss.png}
		\caption{1\_Sigmoid\_EuclideanLoss 对比 test loss}
	\end{subfigure}
	\begin{subfigure}{0.475\textwidth}
		\centering
		\includegraphics[width=1\textwidth]{../pics/学习率_1_Sigmoid_EuclideanLoss_train_loss.png}
		\caption{1\_Sigmoid\_EuclideanLoss 对比 train loss}
	\end{subfigure}
	\begin{subfigure}{0.475\textwidth}
		\centering
		\includegraphics[width=1\textwidth]{../pics/学习率_1_Sigmoid_EuclideanLoss_test_acc.png}
		\caption{1\_Sigmoid\_EuclideanLoss 对比 test accuracy}
	\end{subfigure}
	\begin{subfigure}{0.475\textwidth}
		\centering
		\includegraphics[width=1\textwidth]{../pics/学习率_1_Sigmoid_EuclideanLoss_train_acc.png}
		\caption{1\_Sigmoid\_EuclideanLoss 对比 train accuracy}
	\end{subfigure}
	\caption{基于 1\_Sigmoid\_EuclideanLoss 调节学习率}
	\label{fig:7}
\end{figure}

\begin{figure}[htbp]
	\centering
	\begin{subfigure}{0.475\textwidth}
		\centering
		\includegraphics[width=1\textwidth]{../pics/学习率_2_Sigmoid_Softmax_test_loss.png}
		\caption{2\_Sigmoid\_Softmax 对比 test loss}
	\end{subfigure}
	\begin{subfigure}{0.475\textwidth}
		\centering
		\includegraphics[width=1\textwidth]{../pics/学习率_2_Sigmoid_Softmax_train_loss.png}
		\caption{2\_Sigmoid\_Softmax 对比 train loss}
	\end{subfigure}
	\begin{subfigure}{0.475\textwidth}
		\centering
		\includegraphics[width=1\textwidth]{../pics/学习率_2_Sigmoid_Softmax_test_acc.png}
		\caption{2\_Sigmoid\_Softmax 对比 test accuracy}
	\end{subfigure}
	\begin{subfigure}{0.475\textwidth}
		\centering
		\includegraphics[width=1\textwidth]{../pics/学习率_2_Sigmoid_Softmax_train_acc.png}
		\caption{2\_Sigmoid\_Softmax 对比 train accuracy}
	\end{subfigure}
	\caption{基于 2\_Sigmoid\_Softmax 调节学习率}
	\label{fig:8}
\end{figure}

基于深度学习领域的长期共识,学习率的调整需要基于对于模型本身结构的认知,并没有普适的优秀学习率能够提供普遍优秀的学习效果;譬如大多数模型适用的 1e-2 学习率用于 2\_Sigmoid\_Softmax 和 2\_Sigmoid\_Euclidean 的训练则会完全失败。总体上,对于梯度消失非常严重的双层 Sigmoid 激活函数模型,需要采用较大的学习率,而基于其他激活函数的大多模型能够依靠 1e-2 成功训练。
\\
对于 1\_Sigmoid\_EuclideanLoss 而言可以发现,学习率低于 1e-2 后,随着学习率的减小模型在测试集上的表现会显著降低。观察训练过程的准确率曲线可以发现,随着学习率的增大,参数更新的幅度大幅增大,模型的收敛速度也会变快;然而,模型的表现会出现较大的变动,这在学习率为 1 与 1e-1 的对比上体现的尤为明显。直观上讲,过高的 learning rate 会导致训练很不稳定,同时模型将在多个局部最优值之间反复跳跃。可以猜测,大于 1e-1 后,更大的学习率可能会导致模型难以接近最优的收敛解,最终模型在测试集上的表现也会降低;而较小的学习率却会导致收敛速度过慢以至于在设定的 epoch 次数上训练难以达到最优解,最终在有限的训练成本下,模型在测试集上的表现也会降低。
\\最后,由于本次实验任务较为简单等缘故,实验并没有出现过拟合的情况,而在学习率较低时体现出欠拟合的情况。

\subsection{消减率影响}

保留除了学习率和 weight decay 之外的基础实验参数设定,模型选取 2\_Gelu\_HingeLoss 以 1e-2 为学习率(图 \ref{fig:9})与 2\_Sigmoid\_Softmax 以 1 为学习率(图 \ref{fig:10}),消减率选取为 \{1e-4, 1e-3, 1e-2, 1e-1, 1\},展开消减率对比实验。

\begin{figure}[htbp]
	\centering
	\begin{subfigure}{0.475\textwidth}
		\centering
		\includegraphics[width=1\textwidth]{../pics/消减率_2_Gelu_HingeLoss_test_loss.png}
		\caption{2\_Gelu\_HingeLoss 对比 test loss}
	\end{subfigure}
	\begin{subfigure}{0.475\textwidth}
		\centering
		\includegraphics[width=1\textwidth]{../pics/消减率_2_Gelu_HingeLoss_train_loss.png}
		\caption{2\_Gelu\_HingeLoss 对比 train loss}
	\end{subfigure}
	\begin{subfigure}{0.475\textwidth}
		\centering
		\includegraphics[width=1\textwidth]{../pics/消减率_2_Gelu_HingeLoss_test_acc.png}
		\caption{2\_Gelu\_HingeLoss 对比 test accuracy}
	\end{subfigure}
	\begin{subfigure}{0.475\textwidth}
		\centering
		\includegraphics[width=1\textwidth]{../pics/消减率_2_Gelu_HingeLoss_train_acc.png}
		\caption{2\_Gelu\_HingeLoss  对比 train accuracy}
	\end{subfigure}
	\caption{基于 2\_Gelu\_HingeLoss  调节消减率}
	\label{fig:9}
\end{figure}

\begin{figure}[htbp]
	\centering
	\begin{subfigure}{0.475\textwidth}
		\centering
		\includegraphics[width=1\textwidth]{../pics/消减率_2_Sigmoid_SoftmaxCross_test_loss.png}
		\caption{2\_Sigmoid\_Softmax 对比 test loss}
	\end{subfigure}
	\begin{subfigure}{0.475\textwidth}
		\centering
		\includegraphics[width=1\textwidth]{../pics/消减率_2_Sigmoid_SoftmaxCross_train_loss.png}
		\caption{2\_Sigmoid\_Softmax 对比 train loss}
	\end{subfigure}
	\begin{subfigure}{0.475\textwidth}
		\centering
		\includegraphics[width=1\textwidth]{../pics/消减率_2_Sigmoid_SoftmaxCross_test_acc.png}
		\caption{2\_Sigmoid\_Softmax 对比 test accuracy}
	\end{subfigure}
	\begin{subfigure}{0.475\textwidth}
		\centering
		\includegraphics[width=1\textwidth]{../pics/消减率_2_Sigmoid_SoftmaxCross_train_acc.png}
		\caption{2\_Sigmoid\_Softmax 对比 train accuracy}
	\end{subfigure}
	\caption{基于 2\_Sigmoid\_Softmax 调节消减率}
	\label{fig:10}
\end{figure}

一般而言,权值消减(weight decay)的使用并不是为了提高准确率度也不是为了提高收敛速度,而是最终目的是防止过拟合。在损失函数中,weight decay 放置于正则项(regularization)前作为系数。一般而言,正则项表征模型的复杂度,weight decay 用于调节模型复杂度对损失函数的影响;若 weight decay 很大,则复杂的模型损失函数的值也就大,也即模型会倾向于减少参数值之和以避免过拟合。

从另一方面来讲,当过拟合现象并不明显时,提高 weight decay 反而会抑制降低模型的表达能力。考虑到本次作业之前的数个实验并未出现显著的过拟合现象,故而提高 weight decay 更有可能导致模型在有限的训练时间里失去表达能力。具体到 2\_Gelu\_HingeLoss 与 2\_Sigmoid\_Softmax 模型而言,本身在各自最优学习率的前提下,提高 weight decay 果然降低了模型的表达能力。可以观察到,模型的表达能力和 weight decay 基本负相关,当 weight decay 为 1 时,两个模型在测试集上的准确率均在 0.105 左右波动;对于十分类问题,这等价随机输出结果,符合期望。

\label{paragraph:1}

\subsection{批量大小影响}

模型选取 1\_Sigmoid\_EuclideanLoss 以 1e-2 为学习率(图 \ref{fig:11})与 2\_Sigmoid\_Softmax 以 1 为学习率(图 \ref{fig:12}),消减率均选取为 0。以\{10, 20, 50, 100, 200\} 为批量大小,展开批量大小对比实验。

\begin{figure}[htbp]
	\centering
	\begin{subfigure}{0.475\textwidth}
		\centering
		\includegraphics[width=1\textwidth]{../pics/批量_1_Sigmoid_EuclideanLoss_test_loss.png}
		\caption{1\_Sigmoid\_EuclideanLoss 对比 test loss}
	\end{subfigure}
	\begin{subfigure}{0.475\textwidth}
		\centering
		\includegraphics[width=1\textwidth]{../pics/批量_1_Sigmoid_EuclideanLoss_train_loss.png}
		\caption{1\_Sigmoid\_EuclideanLoss 对比 train loss}
	\end{subfigure}
	\begin{subfigure}{0.475\textwidth}
		\centering
		\includegraphics[width=1\textwidth]{../pics/批量_1_Sigmoid_EuclideanLoss_test_acc.png}
		\caption{1\_Sigmoid\_EuclideanLoss 对比 test accuracy}
	\end{subfigure}
	\begin{subfigure}{0.475\textwidth}
		\centering
		\includegraphics[width=1\textwidth]{../pics/批量_1_Sigmoid_EuclideanLoss_train_acc.png}
		\caption{1\_Sigmoid\_EuclideanLoss  对比 train accuracy}
	\end{subfigure}
	\caption{基于 1\_Sigmoid\_EuclideanLoss  调节批量大小}
	\label{fig:11}
\end{figure}

\begin{figure}[htbp]
	\centering
	\begin{subfigure}{0.475\textwidth}
		\centering
		\includegraphics[width=1\textwidth]{../pics/批量大小_2_Sigmoid_Softmax_test_loss.png}
		\caption{2\_Sigmoid\_Softmax 对比 test loss}
	\end{subfigure}
	\begin{subfigure}{0.475\textwidth}
		\centering
		\includegraphics[width=1\textwidth]{../pics/批量大小_2_Sigmoid_Softmax_train_loss.png}
		\caption{2\_Sigmoid\_Softmax 对比 train loss}
	\end{subfigure}
	\begin{subfigure}{0.475\textwidth}
		\centering
		\includegraphics[width=1\textwidth]{../pics/批量大小_2_Sigmoid_Softmax_test_acc.png}
		\caption{2\_Sigmoid\_Softmax 对比 test accuracy}
	\end{subfigure}
	\begin{subfigure}{0.475\textwidth}
		\centering
		\includegraphics[width=1\textwidth]{../pics/批量大小_2_Sigmoid_Softmax_train_acc.png}
		\caption{2\_Sigmoid\_Softmax 对比 train accuracy}
	\end{subfigure}
	\caption{基于 2\_Sigmoid\_Softmax 调节批量大小}
	\label{fig:12}
\end{figure}

由于选取的模型基本实验结果已经接近完全正确,故而在 50 epoch 下调整批量大小并未发现显著的差距,故而首先选择 1\_Sigmoid\_EuclideanLoss 以 1e-2 为学习率,
放大前 300 step,并且集中对比 10, 200 为批量大小的实验结果;其次选择 2\_Sigmoid\_Softmax 以 1 为学习率,放大前 250 step,并且同样集中对比 10, 200 为批量大小的实验结果。
其结果如图 \ref{fig:13a} 和图 \ref{fig:13b} 所示。

\begin{figure}[htbp]
	\flushleft
	\centering
	\begin{subfigure}{0.475\textwidth}
		\centering
		\includegraphics[width=1\textwidth]{../pics/学习率放大1.jpg}
		\caption{放大前 300 step,观察 1\_Sigmoid\_Euclidean 模型}
		\label{fig:13a}
	\end{subfigure}
	\begin{subfigure}{0.475\textwidth}
		\centering
		\includegraphics[width=1\textwidth]{../pics/学习率放大2.jpg}
		\caption{放大前 250 step,观察 2\_Sigmoid\_Softmax 模型}
		\label{fig:13b}
	\end{subfigure}
	\caption{放大调节批量大小的前 300 step 和 250 step 并进行对比}
	\label{fig:13}
\end{figure}

一般而言,batch size 越大,批次越少,训练时间会更快一点,但可能造成数据的大量浪费;而 batch size 越小,对数据的利用越充分,浪费的数据量越少,但批次较大,训练会更耗时。
如两图所示,batch size 更大时,能够考虑到更全局的信息,训练过程更加稳定,学习效果的波动更小;而较小的 batch size 梯度更新的频率更高,更容易引入样本噪声,训练效果相应波动性更大。不过,
较小的 batch size 引入样本噪声并不一定绝对有害,实际上在 Gau-GAN 的训练过程中,小的 batch size 提升了生成器的泛化效果。\\
最后参考 batch size 对于训练时间的影响。2\_Sigmoid\_SoftmaxCrossEntropyLoss\_1.0\_0.0\_200 用时 8m 42s,而 2\_Sigmoid\_SoftmaxCrossEntropyLoss\_1.0\_0.0\_10 用时 10m 24s;
1\_Sigmoid\_EuclideanLoss\_0.01\_0.0\_10 用时 8m 0s,而 1\_Sigmoid\_EuclideanLoss\_0.01\_0.0\_200 用时 7m 35s,符合之前的预期。
\end{document}